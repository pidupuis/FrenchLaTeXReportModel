
\usepackage[frenchb, english]{babel} % Langue : frenchb pour pouvoir ecrire '2ème' par exemple, et english pour caler des anglicismes.
\usepackage[utf8]{inputenc} % On choisit l'encodage (alterner entre utf8 et latin1 si ca marche pas).
\usepackage[T1]{fontenc} % Permet utilisation nouvelle norme LaTeX.

\usepackage[top=2.5cm, bottom=3cm, left=3.5cm, right=2.5cm]{geometry} % On definit les marges du document.

%% INFOS
% COMMUNES
\title{\textbf{Master 1\iere{} Année} \\ \textsc{Rapport de stage}} % Titre du document
\formation{Master $\LaTeX$, paperasse et administration} % Formation
\mention{Mention Origami et cube impossible d'Escher} % Mention de la formation
\emphasis{Spécialité Chèvres \& coquelicots} % Specialite de la formation
% SPECIFIQUES
\subtitle{A french $\LaTeX$ report model for the glory, for the King and also for a small package of Chocobon.} % Sujet du stage
\author{Présenté par : \textsc{Pierre DUPUIS}} % Auteur
\supervisor{Supervisé par : \textsc{Lui-même}} % Superviseur
\dates{01/06/2013 -- 31/08/2013} % Dates de stage
\address{13, Boulevard of broken dream \\ XXXXX Inconito -- Anonymat} % Adresse

%% LOGOS
% COMMUNS
\llogo{img/autruche.jpg}{0.15}% Logo left
\rlogo{img/autruche.jpg}{0.15}% Logo right
\frlogo{img/autruche.jpg}{0.15}% Logo foot right
% SPECIFIQUES
\fllogo{img/autruche.jpg}{0.15}% Logo foot left
